\usepackage{lmodern}
\usepackage{fontenc}[t1]
\usepackage{microtype}
\usepackage{caption}
\usepackage{enumerate}
\usepackage{hyperref}
\usepackage{wrapfig}
\usepackage{xifthen}
\usepackage{booktabs} % nicer tabular lines
\usepackage{spreadtab} % spreadsheet functionality
\usepackage{ltablex} % big table

\usepackage{array}   
\newcolumntype{L}{>{$}l<{$}} 
\newcolumntype{C}{>{$}c<{$}} 
\newcolumntype{R}{>{$}r<{$}} 

\usepackage[framemethod=TikZ]{mdframed}
\usepackage[customcolors,norndcorners]{hf-tikz}


% code listings
% Default fixed font does not support bold face
\DeclareFixedFont{\ttb}{T1}{txtt}{bx}{n}{12} % for bold
\DeclareFixedFont{\ttm}{T1}{txtt}{m}{n}{12}  % for normal

% Custom colors
\usepackage{color}
\definecolor{deepblue}{rgb}{0,0,0.5}
\definecolor{deepred}{rgb}{0.6,0,0}
\definecolor{deepgreen}{rgb}{0,0.5,0}

\usepackage{listings}

% Python style for highlighting
\newcommand\pythonstyle{\lstset{
language=Python,
basicstyle=\ttm,
morekeywords={self},              % Add keywords here
keywordstyle=\ttb\color{deepblue},
emph={MyClass,__init__},          % Custom highlighting
emphstyle=\ttb\color{deepred},    % Custom highlighting style
stringstyle=\color{deepgreen},
frame=tb,                         % Any extra options here
showstringspaces=false
}}
\usepackage{subcaption}
\DeclareCaptionFormat{custom}
{%
  %\textbf{#1#2}
  {\small #3}
}
\captionsetup{format=custom}

% Python environment
\lstnewenvironment{python}[1][]
{
\pythonstyle
\lstset{#1}
}
{}

% Python for external files
\newcommand\pythonexternal[2][]{{
\pythonstyle
\lstinputlisting[#1]{#2}}}

% Python for inline
\newcommand\pythoninline[1]{{\pythonstyle\lstinline!#1!}}
\usepackage{geometry}
\usepackage{fancyhdr}
  \pagestyle{fancy}
  \fancyhf{}
  \hoffset = -0.45 in
  \voffset = -0.3 in
  \textwidth = 500pt
  \textheight = 650pt
  \setlength{\headheight}{25.6pt}
  \setlength{\headwidth}{500pt}
  %\setlength{\parindent}{0pt}
  \setcounter{secnumdepth}{0}
  \marginparwidth = 0pt
  \fancyhead[L]{\leftmark}
  \fancyhead[R]{\rightmark}
  \fancyfoot[C]{\thepage}
\usepackage{amsmath}
\usepackage{amssymb} 
\usepackage{amsthm} 
\usepackage{dsfont}
\usepackage{amsfonts}
\usepackage{mathdots}
\usepackage{mathrsfs}
\usepackage{bm} % bold math
\usepackage{mathtools}
\usepackage{faktor} % Quotients
\usepackage{xcolor}
\usepackage{stmaryrd}
\usepackage{prftree} % proof tree

\let\tmpphi\phi 
\let\phi\varphi
\let\varphi\tmpphi 
\usepackage{graphicx}
\usepackage[all]{xy} % xypic
  \SelectTips{cm}{}
  \SilentMatrices

\usepackage{tikz}
  \pgfdeclarelayer{edgelayer}
  \pgfdeclarelayer{nodelayer}
  \pgfsetlayers{edgelayer,nodelayer}
  \usetikzlibrary{cd,shapes,decorations,arrows,calc,arrows.meta,fit,positioning,fit}
  %\usetikzlibrary{arrows.meta}

\usepackage{sectsty}
    \sectionfont{\LARGE}
    \subsectionfont{\Large}
    \subsubsectionfont{\large}
    \paragraphfont{\large}

\usepackage[export]{adjustbox}  %\adjustbox{scale=2,center}{}, export allows vertical alignment in includegraphics
\usepackage{paracol}
\usepackage{float}
\usepackage{xcolor}
\definecolor{myblue}{HTML}{81a1c1}
\definecolor{mygreen}{HTML}{a3be8c}
\newcommand{\N}{\mathbb{N}}
\newcommand{\Z}{\mathbb{Z}}
\newcommand{\Q}{\mathbb{Q}}
\newcommand{\R}{\mathbb{R}}
\newcommand{\IS}{\mathbb{S}}
\newcommand{\ID}{\mathbb{D}}
\newcommand{\IT}{\mathbb{T}}
\newcommand{\C}{\mathbb{C}}
\newcommand{\K}{\mathbb{K}}
\newcommand{\F}{\mathbb{F}}
\newcommand{\IP}{\mathbb{P}}
\newcommand{\RP}{\mathbb{RP}}
\newcommand{\from}{\leftarrow}


\DeclareMathOperator{\charac}{char}
\DeclareMathOperator{\degree}{deg}  

\newcommand{\del}{\partial}
\newcommand{\es}{\text{\o}}
\newcommand{\rep}{rel.\ endpoints}
\DeclarePairedDelimiter\abs{\vert}{\vert}
\DeclarePairedDelimiter\ceil{\lceil}{\rceil}
\DeclarePairedDelimiter\floor{\lfloor}{\rfloor}
\DeclarePairedDelimiter\Norm{\vert\vert}{\vert\vert}
\DeclarePairedDelimiter\scal{\langle}{\rangle}
\DeclarePairedDelimiter\dbrack{\llbracket}{\rrbracket}
% LinAlg
\DeclareMathOperator{\Hom}{Hom}
\DeclareMathOperator{\Ker}{Ker}
\DeclareMathOperator{\Image}{Im}
\DeclareMathOperator{\image}{im}
\DeclareMathOperator{\id}{id}
\DeclareMathOperator{\End}{End}
\DeclareMathOperator{\ev}{ev}
\DeclareMathOperator{\coev}{coev}
\DeclareMathOperator{\trace}{tr}
\DeclareMathOperator{\Mat}{Mat}
\DeclareMathOperator{\GL}{GL}

% Algebra
\DeclareMathOperator{\lcd}{lcd}
\DeclareMathOperator{\ggT}{ggT}
\DeclareMathOperator{\Aut}{Aut}
\DeclareMathOperator{\sgn}{sgn}

% Topology
\DeclareMathOperator{\const}{const}
\DeclareMathOperator{\Deck}{Deck}

% Category
\newcommand*{\Top}{\text{\sffamily Top}}
\newcommand*{\Htpy}{\text{\sffamily Htpy}}
\newcommand*{\Grp}{\text{\sffamily Grp}}
\newcommand*{\Set}{\text{\sffamily Set}}
\newcommand*{\Vect}{\text{\sffamily Vect}}
\newcommand*{\Rel}{\text{\sffamily Rel}}
\newcommand*{\Ring}{\text{\sffamily Ring}}

\newcommand{\iso}{\cong}
\newcommand{\mono}{\rightarrowtail}
\newcommand{\epi}{\twoheadrightarrow}
\renewcommand{\labelenumi}{(\alph{enumi})}
\usepackage{comment}
\ifnum\showsolutions=1
  \newenvironment{sol}{\textbf{Solution:}~}{}
\else
  \excludecomment{sol}
\fi
\newcounter{theo}[subsection]
\newcounter{exer}[section]
\renewcommand{\thetheo}{\arabic{section}.\arabic{subsection}.\arabic{theo}}
\renewcommand{\theexer}{\arabic{section}.\arabic{exer}}

\newcommand{\defboxenv}[5]{
  \newenvironment{#1}[1][]{
    \refstepcounter{#4}
    \ifstrempty{##1} { % no Title
      \mdfsetup{
        frametitle={
          \tikz[baseline=(current bounding box.east), outer sep=0pt]
          \node[anchor=east,rectangle,fill=#3]
          {\strut #2~#5};
        }
      }
    }{ % else: with Title
      \mdfsetup{
        frametitle={
          \tikz[baseline=(current bounding box.east),outer sep=0pt]
          \node[anchor=east,rectangle,fill=#3]
          {\strut #2~#5:~##1};
        }
      }
    } % either case
    \mdfsetup{
      linecolor=#3,
      innertopmargin=0pt,
      linewidth=2pt,
      topline=true,
      frametitleaboveskip=\dimexpr-\ht\strutbox\relax,
    }
    \begin{mdframed}[]\relax
    }{ % post env command
    \end{mdframed}
  }
}
\defboxenv{dfn}{Definition}{orange!50}{theo}{\thetheo}
\defboxenv{edfn}{(Definition)}{orange!20}{theo}{\thetheo}
\defboxenv{thm}{Theorem}{blue!40}{theo}{\thetheo}
\defboxenv{lem}{Lemma}{blue!20}{theo}{\thetheo}
\defboxenv{prop}{Proposition}{myblue}{theo}{\thetheo}
\defboxenv{cor}{Corollary}{blue!30}{theo}{\thetheo}
\defboxenv{xmp}{Example}{green!50!blue!30!white}{theo}{\thetheo}
\defboxenv{rem}{Remark}{red!20}{theo}{\thetheo}
\defboxenv{exr}{Exercise}{black!30!white}{exer}{\theexer}

% ACT4E
\newcommand\lList{\scalebox{0.3}[1.2]{$\lhd$}}
\newcommand\rList{\scalebox{0.3}[1.2]{$\rhd$}}
\newcommand\asList[1]{\lList #1\rList}
\newcommand\SetL{\asList{\textsf{Set}}}
