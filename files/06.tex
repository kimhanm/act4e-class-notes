\section{Week 06}
On the Week 03 exercises, I had the wrong number of exercises listed in the exercise class notes. The exercise \texttt{ProductPoset} was not published, so the max number of points for that week is 18 instead of 23.

\subsection{Commutative Diagrams}
Some mathematical statements take a lot words to write down.
Commutative diagrams let us organize and condense that information in a neat way.

A \textbf{triangle diagram} in a category $\textsf{C}$ consists of three objects $X,Y,Z$ and three morphisms $f,g,h$ in the form
\begin{center}
\begin{tikzcd}[column sep=0.8em] 
  &Y \arrow[]{dr}{g}
  \\
  X
  \arrow[]{ur}{f}
  \arrow[]{rr}{h}
  &&
  Z
\end{tikzcd}
\end{center}
The triangle diagram \textbf{commutes}, if $f;g = h$.

A (general) \textbf{diagram} in $\textsf{C}$ is any collection of objects and morphisms between them.
A diagram \textbf{commutes}, if all of its induced triangles commute.

\textbf{Note:} Triangles need not be triangular.
For example, in this diagram
\begin{center}
\begin{tikzcd}[] 
  A 
  \arrow[]{r}{f}
  \arrow[]{d}{\alpha}
  & B 
  \arrow[]{r}{g}
  \arrow[]{d}{\beta}
  & C
  \arrow[]{d}{\gamma}
  \\
  X 
  \arrow[]{r}{p}
  & Y 
  \arrow[]{r}{q}
  & Z
\end{tikzcd}
\end{center}
There are no triangular triangles, but since we ca compose morphisms, the following forms a triangle:
\begin{center}
\begin{tikzcd}[ ] 
  A 
  \arrow[]{rr}{f;g}
  \arrow[]{drr}{\alpha;p;q}
  & &
  \arrow[]{d}{\gamma}
  C
  \\
  & & Z
\end{tikzcd}
\end{center}
so on order for the entire diagram to commute, this triangle must commute, which means $f;g;\gamma = \alpha;p;q$.

\textbf{Warning:} Don't forget the identity morphism. For example
\begin{align*}
  \xymatrix{
    X
    \ar@/^2ex/[r]^{f}
    &
    Y
    \ar@/^2ex/[l]^{g}
  }
\quad \text{should be thought of as} \quad
  \xymatrix{
    X
    \ar@(dl,ul)[]^{\id_X}
    \ar@/^2ex/[r]^{f}
    &
    Y
    \ar@/^2ex/[l]^{g}
    \ar@(ur,dr)[]^{\id_Y}
  }
\end{align*}
Which includes two important triangles
\begin{center}
\begin{tikzcd}[column sep=0.8em] 
  &Y
  \arrow[]{dr}{g}
  \\
  X
  \arrow[]{ur}{f}
  \arrow[]{rr}{\id_X}
  &&X
\end{tikzcd}
\quad \text{and} \quad 
\begin{tikzcd}[column sep=0.8em] 
  &X
  \arrow[]{dr}{f}
  \\
  Y
  \arrow[]{ur}{g}
  \arrow[]{rr}{\id_Y}
  &&Y
\end{tikzcd}
\end{center}




To give an example of how much space this saves, draw the commutative diagram for the statement
\begin{center}
  Let $T,L,A_1,\ldots,A_5$ be objects and $\psi: T \to L, g_i: T \to A_i, f_i: L \to A_i$ for $i = 1,\ldots,5$ and $a_i: A_i \to A_{i+1}$ for $i = 1,\ldots 4$ 
  morphisms such that 
  $\psi;f_i = g_i$ for $i = 1,\ldots, 5$ and 
  \begin{align}
  \psi;f_i = g_i \quad \text{and} \quad 
  i = 1,\ldots, 5,
  \quad \text{and} \quad 
  g_i;a_i = g_{i+1}
  ,\quad
  f_i;a_i = f_{i+1}
  \quad \text{for} \quad 
  i = 1,\ldots, 4
  \label{eq:conditions}
  \end{align}
\end{center}

\textbf{Solution:}
\begin{align*}
  \xymatrix{
    &&
    T
    \ar[d]|{\psi}
    \ar[ddll]|{g_1}
    \ar[ddl]|{g_2}
    \ar@/^2ex/[dd]|{g_3}
    \ar[ddr]|{g_4}
    \ar[ddrr]|{g_5}
    \\
    &&
    L
    \ar[dll]|{f_1}
    \ar[dl]|{f_2}
    \ar[d]|{f_3}
    \ar[dr]|{f_4}
    \ar[drr]|{f_5}
    \\
    A_1
    \ar[r]|{a_1}
    &
    A_2
    \ar[r]|{a_2}
    &
    A_3
    \ar[r]|{a_3}
    &
    A_4
    \ar[r]|{a_4}
    &
    A_5
  }
\end{align*}
Can you see all the triangles in this diagram?

\textbf{Note:} If in a diagram, all triangles commute, then \emph{any two paths} in the diagram are the same.

As an exercise, show using just Equations~\eqref{eq:conditions} that 
\begin{align*}
  g_5 = \psi;f_1;a_1;a_2;a_3;a_4
\end{align*}

\textbf{Solution:}
\begin{align*}
  g_5 = \psi;f_5 = \psi;f_4;a_4 = \psi;f_3;a_3;a_4
  = \psi;f_2;a_2;a_3;a_4
  = \psi;f_1;a_1;a_2;a_3;a_4
\end{align*}

Fun fact: This particular diagram (or rather, a generalization of it) is used in algebra.
If we write $\Z/n\Z = \{0,1,\ldots,n-1\}$, then the \emph{$10$-adic integers} $\Z_{10}$ can represent ``numbers'' like $\cdots 111111$ or $\cdots99999 = -1$.

for the exercise \texttt{InventingCommDiagrams}, I suggest that you keep the diagrams small, so do not go bigger than the example given.

\subsection{Categorical Constructions}
\subsubsection{Slice Categories}
If $\textsf{C}$ is a category and $c \in \text{Ob}(\textsf{C})$.
Then the \textbf{slice category} of $c$ \textbf{over} $\textsf{C}$ is the category $\textsf{C}/c$, where
\begin{enumerate}
  \item The objects are morphisms $x \to c$ in $\textsf{C}$
  \item A morphism of between the objects $(x \stackrel{f}{\to}c)$ and $(y \stackrel{g}{\to}c)$ is a morphism $x \stackrel{\phi}{\to} y$ such that the following diagram commutes:
    \begin{center}
    \begin{tikzcd}[column sep=0.8em] 
      x \arrow[]{rr}{\phi} 
      \arrow[swap]{dr}{f}
      && y
      \arrow[]{dl}{g}
      \\
      & c 
    \end{tikzcd}
    \end{center}
  \item The composition of morphisms
    \begin{center}
    \begin{tikzcd}[column sep=0.8em] 
      x \arrow[]{rr}{\phi} 
      \arrow[swap]{dr}{f}
      && y
      \arrow[]{dl}{g}
      \\
      & c 
    \end{tikzcd}
    \begin{tikzcd}[column sep=0.8em] 
      y \arrow[]{rr}{\psi} 
      \arrow[swap]{dr}{g}
      && z
      \arrow[]{dl}{h}
      \\
      & c 
    \end{tikzcd}
    \end{center}
    is the $\textsf{C}$-morphism $\phi;\psi$.

    \textbf{In the exercise, you need to show that $\phi;\psi$ is a valid morphism in $\textsf{C}/c$.}
\end{enumerate}


Here is an example of a slice category:

If $(P,\leq)$ is a poset, then the slice category $\mathcal{P}/a$ can be viewed as consisting of elements which are $\leq a$.
The condition in the defintion (Point b) is always satisfied, because \emph{all} diagrams in $\mathcal{P}$ commute. Why?
    
For example, in the poset $(\N,\leq)$ the slice category $\mathcal{N}/5$ consists of numbers $\{0,1,2,3,4,5\}$. (We are \emph{slicing} off all numbers after $5$.)

%Consider the category $\textsf{C}$ where the objects are 
%\begin{align*}
%  \underbrace{\emptyset}_{=: 0}
%  , \quad
%  \underbrace{\{x\}}_{=: 1}
%  , \quad 
%  \underbrace{\{x,y\}}_{=: 2}
%  , \quad 
%  \underbrace{\{x,y,z\}}_{=: 3}
%\end{align*}
%and the morphisms are functions
%\begin{itemize}
%  \item $\textsf{C}/0$ has one object, namely $(\emptyset \stackrel{\emptyset \times \emptyset}{\to} \emptyset)$. Why?
%  \item $\textsf{C}/1$ looks exactly like $\textsf{C}$.
%  \item To figure out $\textsf{C}/2$, let's look at its objects. We can identify a function $2 \to \{x,y\}$ as a sequence $xx,xy,yx,yy$ of length $2$. Likewise, we can identify a function $3 \to \{x,y\}$ as a sequence $xxx,xxxy,xyx,\ldots$ of length $3$.
%    Thus, the objects of $\textsf{C}/2$ are
%    \begin{align*}
%      \_: 0 \to 2
%      , \quad
%      x,y: 1 \to 2
%      , \quad
%      xx,xy,yx,yy: 2 \to 2
%      , \quad
%      xxx,xxy,xyx,xyy,yxx,yxy,yyx,yyy: 3 \to 2
%    \end{align*}
%    Now, let's look at the Hom-sets $\Hom_{\textsf{C}/2}(a,b)$. We consider the cases in $a$:
%    \textbf{$\bm{a = \_}$:} 
%    $\Hom_{\textsf{C}/2}(\_,b)$ consists of functions with source $0$, of which there is exactly one.
%    \begin{align*}
%      \Hom_{\textsf{C}/2}(\_,b) = \{\ast\}
%    \end{align*}
%    \textbf{$\bm{a = x}$:} 
%\end{itemize}


\subsubsection{Twisted Arrow Categories}

Consider the Poset $(\R,\leq)$. Our goal is to understand the twisted arrow category $\texttt{Tw }\mathcal{R}$.

\begin{enumerate}
  \item Its objects are morphisms $a \to b$ in the category $\mathcal{R}$, which are pairs of real numbers such that $a \leq b$.
    Now, we identify the objects of $\texttt{Tw }\mathcal{R}$ with the closed intervals $[a,b] \subseteq \R$.
  \item A morphism is then a diagram of the form
    \begin{align*}
      \xymatrix{
        a
        \ar[d]|{\leq}
        &
        c
        \ar[l]|{\geq}
        \ar[d]|{\leq}
        \\
        b
        \ar[r]|{\leq}
        &
        d
      }
    \end{align*}
    so we say that an interval $[a,b]$ has a morphism to an interval $[c,d]$ if and only if $c \leq a$ and $b \leq d$.
    If you draw this on the real line, this precisely means that the interval $[a,b]$ is contained in $[c,d]$.
  \item Thus, we can view the twisted arrow category $\texttt{Tw }\mathcal{R}$ as the poset of closed intervals ordered by inclusion.
\end{enumerate}


\subsubsection{The Categorical Product}
In a category $\textsf{C}$, a \textbf{cone} over objects $A,B$ is an object $X$ of $\textsf{C}$ equipped with to morphisms

\begin{center}
\begin{tikzcd}[ ] 
  A
  &
  X
  \arrow[swap]{l}{f}
  \arrow[]{r}{g}
  &
  B
\end{tikzcd}
\end{center}
A cone $A \stackrel{\text{pr}_A}{\from} X \stackrel{\text{pr}_B}{\to} B$ is called \textbf{product} of $A,B$, if it is \emph{universal} in the sense that:

For any other cone $A \stackrel{f}{\from}T \stackrel{g}{\to}B$, there exists a \emph{unique} morphism $\psi: T \to X$, such that the following diagram commutes:

\begin{center}
\begin{tikzcd}[] 
  &T
  \arrow[bend right,swap]{ddl}{f}
  \arrow[bend left]{ddr}{g}
  \arrow[dotted]{d}{\psi}
  \\
  &X
  \arrow[]{dl}{\text{pr}_A}
  \arrow[swap]{dr}{\text{pr}_B}
  \\
  A && B
\end{tikzcd}
\end{center}

In the following, we will consider an example that illustrates the definition.


Let $A = \{x,y\}$, $B = \{1,2\}$.
Consider the sets 
\begin{align*}
S = \{(x,1),(x,2),(y,1)\}
,\quad \text{and} \quad 
R = \{(x,1),(x,2),(y,1),(y,2),r\}
\end{align*}
which are not isomorphic to $A \times B$. Which of the following is a reason why they can not be made into categorical product of $A$ and $B$?
\begin{enumerate}
  \item there do not exist $S$ morphisms $A\stackrel{\text{pr}_A}{\leftarrow}S \stackrel{\text{pr}_S}{\to}B$.
  \item $S$ has morphisms $A \stackrel{\text{pr}_A}{\leftarrow} S \stackrel{\text{pr}_B}{\to} B$ but for another object $T$ with morphisms $A \stackrel{f}{\leftarrow} T \stackrel{g}{\to}B$, there does not exist a morphism $\phi_{f,g}: T \to S$ such that $f = \phi_{f,g};\text{pr}_A$ and $g = \phi_{f,g};\text{pr}_B$.
  \item $R$ has morphisms $A \stackrel{\text{pr}_A}{\leftarrow} R \stackrel{\text{pr}_B}{\to} B$ but for another object $T$ with morphisms $A \stackrel{f}{\leftarrow} T \stackrel{g}{\to}B$, there does not exist a morphism $\phi_{f,g}: T \to S$ such that $f = \phi_{f,g};\text{pr}_A$ and $g = \phi_{f,g};\text{pr}_B$.
  \item Don't know
\end{enumerate}
\textbf{Solution: (b)} 
Consider the test-object $T = A \times B$ with $f,g$ the projection morphisms. Then $\phi_{f,g}$ cannot correctly map the element $(y,1)$ into $S$ correctly.

$R$ however does have such a morphism $\phi_{f,g}: T \to R$, but it is not a product because $\phi_{f,g}$ is \textbf{not unique}!
If for example $\text{pr}_A(r) = x, \text{pr}_B(r) = 1$, then one can chose for the cone $A \stackrel{f}{\from}A \times B \stackrel{g}{\to}B$ either $\phi_{f,g}(x,1) = (x,1)$ \textbf{or} $\phi_{f,g}(x,1) = r$.

In the category $\Set$: If the cone has too few elements (like $S$), then such a morphism $\phi_{f,g}$ does not always exist.
If the cone has too many elements (like $R$), then $\psi_{f,g}$ does exist, but is not unique.



