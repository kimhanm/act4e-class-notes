\section{Week 09}
Because there was no lecture this week, we will do revisit some material today. 
If you have any questions about some of the exercise sheets, you can ask them and I will include them in the uploaded version on moodle.

\subsection{Unabstraction}
Since the introduction of categories, we have introduced the following concepts:
\begin{itemize}
  \item Commutative Diagrams
  \item Functors
  \item Categorical Product
  \item Opposite Category
  \item Slice Category
\end{itemize}

Today, we will look back at some of the things we did earlier and try to see them as manifestations of the things above.

For each of the following ideas, see if you can describe them using the concepts above.

\begin{enumerate}
  \item The greatest common divisor $\gcd(n,m)$ of two natural numbers $n,m$. 
  \item The transpose $R^{T}$ of a relation $R: X \to Y$ given by $x R y \iff y R^{T}x$. It holds ${(R^{T})}^{T} = R$.
  \item The coproduct.
  \item The coslice category $c/\textsf{C}$
  \item The singleton set.
  \item The empty set.
  \item Surjective and injective functions.
  \item[(h$^{\ast}$)] For a real number $x \in \R$, we can define the flor $\floor{x}$ and ceiling $\ceil{x}$. For example: $\floor{\pi} = 3, \ceil{\pi} = 4$.
    Then
    \begin{align*}
      n \leq 2k + 1
      \quad \text{if and only if} \quad 
      2 \floor{\tfrac{n}{2}} + 1 \leq 2k + 1
      \\
      2k \leq n
      \quad \text{if and only if} \quad 
      2k \leq \floor{\tfrac{n}{2}}
    \end{align*}
    \setcounter{enumi}{8}
  \item The dual of a real (finite-dimensional) vector space $V^{\ast} = \Hom(V,\R)$.
  \item The poset of intervals $[a,b] \subseteq \R$ ordered by inclusion.
  \item The tensor product of vector spaces.
\end{enumerate}


\textbf{Possible Solutions:}
\begin{enumerate}
  \item The $\gcd$ is the categorical product. In the category, where the objects are natural numbers and
    \begin{align*}
      \Hom(n,m) = \{d \in \N \big\vert d \cdot n = m\}
    \end{align*}
    \textbf{TODO:} Show diagram
  \item Transposition gives rise to a contravariant functor $T:\Rel \to \Rel$. The fac that ${(R^{T})}^{T} = R$ can be interpreted as the equality of functors $T;T = \id_{\Rel}$.
  \item The coproduct is the product in the opposite category. If you are not 100\% convinced of this, go through the definition and prove it.
  \item The coslice category is opposite category of the slice category of the opposite category
    \begin{align*}
      c/\textsf{C} = {(\textsf{C}^{\text{op}}/c)}^{\text{op}}
    \end{align*}
  \item The singleton set is the product of an empty colleciton of sets in $\Set$
  \item The empty set is the coproduct of an empty collection of sets in $\Set$.
  \item We can define the notions of epimorphisms vs monomorphisms. Then the monomorphisms are precisely the epimorphisms in the opposite category.
  \item[(h$^{\ast}$)] \textbf{TODO:}

    \setcounter{enumi}{8}
  \item The dual defines a contravariant functor ${(-)}^{\ast}\Vect_{\R} \to \Vect_{\R}$.
    If $F: V \to W$ is a linear map, then how do we have to define $F^{\ast}: W^{\ast} \to  V^{\ast}$?
  \item This is the twisted arrow category of $(\R,\leq)$. See week 06 exercise class notes.
  \item The tensor product defines a functor $\Vect_\R \times \Vect_{\R} \to \Vect_{\R}$. Unpack the condition of functoriality and that of the product category.
\end{enumerate}


\subsection{Definitions of the above}
You should have the definitions of the concepts form category theory in the book. I 



Recall that for a category $\textsf{C}$, we can define its \textbf{opposite category} $\textsf{C}^{\text{op}}$ where
\begin{itemize}
  \item $\text{Ob}(\textsf{C}^{\text{op}}) := \text{Ob}(\textsf{C})$.
  \item $\Hom_{\textsf{C}^{\text{op}}}(X,Y) := \Hom_{\textsf{C}}(Y,X)$
  \item For $f \in \Hom_{\textsf{C}^{\text{op}}}(X,Y)$, $g \in \Hom_{\textsf{C}^{\text{op}}(Y,Z)}$ define their composition
    \begin{align*}
      f;_{\textsf{C}^{\text{op}}}g := g;_{\textsf{C}}f \in \Hom_{\textsf{C}^{\text{op}}}(X,Z) = \Hom_{\textsf{C}}(Z,X)
    \end{align*}
\end{itemize}


%\subsection{Q\&A}
