\section{Week 04}
%\subsection{Theory Section}
\subsection{Ordered sets}
Ordered sets let us conceptualize and formalize our understanding of trade-offs and valuations.

\subsubsection{Warmup Exercise --- Review of Relations}

For each of the following relations, check if they are reflexive, total, symmetric, transitive, irreflexive, asymmetric or antisymmetric.
\begin{enumerate}
  \item $R: \R \to \R$, where $x R y \iff x \leq y$
  \item $R: \R \to \R$, where $x R y \iff x > y$
  \item $R: \N \to \N$, where $x R y \iff \exists z \in \N: z < x$.
  \item Let $X$ be the duck family $\{\text{Scrooge (Mc Duck)}, \text{Donald}, \text{Daisy}, \text{Tick}, \text{Trick}, \text{Track}\}$ with relation $x R y \iff$ $x$ is Nephew of $y$.

    What is the smallest poset $S$ that contains $R$? Describe it in words. 

    Draw the Hasse Diagram of $S$. What is its height, and width?
\end{enumerate}

\textbf{Solution:}
Here are the definitions of the different relation properties. 
\begin{table}[h]
\begin{tabular}{cccc}
  Injective
  &
  Single valued
  &
  Surjective
  &
  Everywhere defined
  \\
  $x Ry \land x R y \implies x = y'$
  &
  $x R y \land x R y' \implies y = y'$
  &
  $\forall y: \exists x: x Ry$
  &
  $\forall x: \exists y: x Ry$
  \\
  \\
  Reflexive
  & Total 
  & Symmetric
  & Transitive
  \\
  always $x Rx$
  &
  $x R y \lor y R x$
  &
  $x R y \iff y R x$
  &
  $x R y \land y R z \implies x R z$
  \\
  \\
  Irreflexive
  & Asymmetric
  & Antisymmetric
  \\
  never $x R x$
  &
  never both $x R y$ or $y Rx$ 
  &
  $x R y \land y R x \implies x = y$
\end{tabular}
\caption{Summary of endorelation properties.
They are to be read with $\forall$ before each variable, i.e. $\forall x: x R x$}
\end{table}

\begin{table}[h]
\centering
\begin{tabular}{c|ccccccccccc}
  Relation 
  & Inj. & Sing.\ val. & Surj. & Ev.\ def. & Refl.\footnote[2]{} & Tot. & Symm. & Trans.\footnote[2]{} & Irrefl. & Asymm. & Antisymm.\footnote[2]{}\\
  \midrule
  (a)
  & $\times$ & $\times$ & $\checkmark$ & $\checkmark$ & $\checkmark$ & $\checkmark$ & $\times$ & $\checkmark$ & $\times$ & $\times$ & $\checkmark$
  \\
  (b)
  & $\times$ & $\times$ & $\checkmark$ & $\checkmark$ & $\times$ & $\times$ & $\times$ & $\checkmark$ & $\checkmark$ & $\checkmark$ & $\checkmark$\footnote[1]{Ex falso absurdum: If the left side of an impliciation $A \implies B$ is always false, then the implication is always true, no matter what $B$ is.}
  \\
  (c)
  & $\times$ & $\times$ & $\checkmark$ & $\times$
  & $\times$ & $\times$ & $\times$ & $\checkmark$ 
  & $\times$ & $\times$ & $\times$
  \\
  (d)
  & $\times$ & $\checkmark$ & $\times$ & $\times$ & $\times$ & $\times$ & $\times$ & $\times$ & $\checkmark$ & $\checkmark$ & $\checkmark$\footnote[1]{}
\end{tabular}
\caption{Solution for warmup Exercise}
\end{table}
Remarks\footnote[1]{Ex falso absurdum: If the left side of an impliciation $A \implies B$ is always false, then the implication is always true, no matter what $B$ is.}\footnote[2]{Reflexivity, Transitivity and Antisymmetry are the 3 properties that a Poset has.}

For (d),  the smallest poset $S$ containing $R$ could be described by:
\begin{align*}
  x \overline{R} y \iff
  \text{$x$ is $y$, a nephew of $y$ or a nephew of a newphew of $y$}
\end{align*}
The height of $S$ is $3$ and the width of $S$ is $4$.


\subsubsection{Product of Orderings (\texttt{Product Poset})}

Given posets $(P,\leq_P), (Q,\leq_Q)$, their \textbf{product} consists of the set $P \times Q$, with product-relation
\begin{align*}
  \underbrace{(p,q) \leq (p',q')}_{\text{New definition}} \iff p \leq_P p' \land q \leq_Q q'
\end{align*}

As an exercise, consider the poset $P = \{\texttt{true},\texttt{false}\}$, where $\texttt{false}\leq \texttt{true}$.
Draw the Hasse diagram of the product poset $P \times P \times P$.


\subsubsection{Order-preserving maps (Morphisms)}
A function $f: (P, \leq_P), (Q,\leq_Q)$ between posets is a \textbf{poset morphism}, if it preserves the ordering, that is:
\begin{align*}
  \text{for all $p,p' \in P$: }
  \quad
  p \leq_P p' \implies f(p) \leq_Q f(p')
\end{align*}


\subsubsection{\texttt{polynomialDivisibility}}
A quick review on divisibility of polynomials:
A polynomial with real coefficients in an expression of the form
\begin{align*}
  a_0 + a_1 X + a_2 X^{2} + \cdots + a_n X^{n}
\end{align*}
where $a_i \in \R$ are real numbers and $n$ is a (finite!) number in $\N$.

Polynomial multiplication is the same as you know from school. For example for
\begin{align*}
  p = \sqrt{2}X - 1, \quad
  q = \sqrt{2}X + 1
\end{align*}
they are multiplied by distributivity. For example:
\begin{align*}
  p \cdot q 
  &= (\sqrt{2}X - 1) (\sqrt{2} X + 1) 
  \\
  &= 
  \sqrt{2} \cdot \sqrt{2} X^{2} + \sqrt{2}X - \sqrt{2}X - 1
  \\
  &= 2 X^{2} -1
\end{align*}
Note: Constants $0,1, \frac{1}{2}, \sqrt{2}$ are also polynomials.



\subsubsection{\texttt{MonotoneMapCheck}}
For the third part, it may help to write out the divisors of $36$. They are $\{1,2,3,4,6,9,12,18,36\}$. Note that the only odd divisors are $1,3,9$.
More generally, what can you say about the divisors of $y$ if $x \leq y$?

