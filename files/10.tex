\section{Week 10}
Informally, a design problem is one where you want to find all viable solutions that optimize \textbf{functionality} and minimize \textbf{requirements}.

\subsection{Design Problems}


To model this mathematically, we give the following definition
A \textbf{design problem} is a tuple $\scal{F,R,d}$, where $F,R$ are posets and $d$ is a monotone map of the form
\begin{align*}
  d: F^{\text{op}} \times R \to_{\text{Pos}} \texttt{Bool}
\end{align*}
\begin{itemize}
  \item The poset $F^{\text{op}}$ describes the possible modes of functionality. In the example of a drone, one could have $F = (\R_{>0},\leq) \times (\R_{>0},\leq)$ describe its top speed and range.
  \item The poset $R$ describes the requirements. For example one could use $R = (\R_{> 0},\leq)$ to describe its cost.
  \item For $d$ to be monotonous, we require that for $(f,r),(f',r') \in F^{\text{op}} \times R$:
    \begin{align*}
      f \geq f' \land r \leq r' \implies d(f,r) \leq d(f',r')
    \end{align*}
    so if a drone $x$ provides higher functionality $f$ at a lower cost $r$, than another drone $x'$, we prefer drone $x$ over $x'$.
\end{itemize}

In the exercise sheet, you will show that any monotone map $g: F \to_{\text{Pos}} R$ induces a design problem where $d_g(f^{\ast},r) = \top$ if and only if $g(f) \leq r$.

In the case of drones as above, one would have a selection of drones $X$. If a drone $x$ provides functionality $f(x)$ and costs $c(x)$, we define $g(f) = \min_{x, f(x) \geq f} c(x)$


\subsection{Upper Sets}
Given a poset $(P,\leq)$, an \textbf{upper set} of $P$ is a subset $X \subseteq P$ such that
\begin{align*}
  x \in X \land y \geq x \implies y \in X
\end{align*}
Write $\mathcal{U}(P)$ to be the set of upper sets of $P$.
There is a function
\begin{align*}
  \uparrow: \texttt{Pow}(P) \to \mathcal{U}(P),
  \quad
  A \mapsto \uparrow(A)
\end{align*}
that returns the smallest upper set containing $A$ (ordered by inclusion).

\textbf{Exercise:} Let $X = {([0,1],\leq)}^{2}$. Draw $\uparrow(A)$ for
\begin{itemize}
  \item $A = \{(\tfrac{1}{2},\tfrac{1}{2})\}$
  \item $A = \{(\tfrac{1}{4},\tfrac{3}{4}),(\tfrac{3}{4},\tfrac{1}{4})\}$
  \item $A = \{(y,x) \big\vert x + y = 1\}$
  \item $A = \{(y,x) \big\vert x^{2} + y^{2} = 1\}$
\end{itemize}

\subsubsection{The category $\textsf{UPos}$}

We can form a category $\textsf{UPos}$ where
\begin{enumerate}
  \item The objects are posets
  \item A morphism from objects $X,Y$ is a monotone map
    \begin{align*}
      f: X \to  \mathcal{U}(Y)
    \end{align*}
  \item Given morphisms $f: X \to Y$, $g: Y \to Z$, we define their composition as monotone map
    \begin{align*}
      (f;g): X \to \mathcal{U}(Z)
      ,\quad
      x \mapsto \bigcup_{y \in f(x)} g(y)
    \end{align*}
  \item The identity morphism is the upper closure operator $\id_X(x) = \uparrow \{x\}$.
\end{enumerate}
As an exercise, show that this does indeed form a category.

\textbf{$\bm{(f;g)}$ is well-defined:} 
We have to show that for all $x \in X$, $(f;g)(x)$ is an upper set of $Z$.
Let $x \in X$ and let $z \in (f;g)(x)$ and $z' \geq z$. By definition of $(f;g)$, there exists a $y \in f(x)$ such that $z \in g(y)$.
Since the target of $g$ is $\mathcal{U}(Z)$, $g(y)$ is an upper set. Which means $z' \in g(y) \subseteq (f;g)(x)$.

\textbf{$\bm{f;g}$ is monotone:} Let $x,x' \in X$ such that $x \leq x'$.
We need to show that $(f;g)(x) \subseteq (f;g)(x')$.
Since $f$ is monotone, it holds $f(x) \subseteq f(x')$, thus 
\begin{align*}
  (f;g)(x) =
  \bigcup_{y \in f(x)} g(y) \subseteq \bigcup_{y \in f(x')} g(y)
  = (f;g)(x')
\end{align*}

Note that if we defined the composition through
\begin{align*}
  (f;g): X \to \mathcal{U}(Z),
  \quad
  x \mapsto \bigcap_{y \in f(x)} g(y)
\end{align*}
then this alternative composition is no longer monotone.


\textbf{Associativity of composition:} Let $f: X \to \mathcal{U}(Y), g: Y \to \mathcal{U}(Z), h: Z \to \mathcal{U}(W)$ be monotone maps. Then for all $x \in X$ it holds
\begin{align*}
  ((f;g);h)(x) 
  &=
  \bigcup_{z \in (f;g)(x)} h(z)
  \\
  &= \bigcup_{y \in f(x)}
  \bigcup_{z \in g(y)}
  h(z)
  \\
  &=
  \bigcup_{y \in f(x)}
  (g;h)(y)
  \\
  &=
  \bigcup_{y \in f(x)}
  (g;h)(y)
  \\
  &=(f;(g;h))(x)
\end{align*}

\textbf{Identity morphism:} Let $f: X \to \mathcal{U}(Y)$ be a monotone map. Then
\begin{align*}
  (\id_Y;f)(x) = \bigcup_{y \in f(x)} \uparrow\{y\} \supseteq f(x)
\end{align*}
Since $f(x)$ is an upper set of $y$, any $y' \in \uparrow\{y\}$ is also an element of $f(x)$.
Likewise one shows $f;\id_X = f$.
